\section{Appendix}
\label{sec:appendix}

\subsection{\venera System Manual}

Venera is a command-line tool that hooks into any Android test runner and
instruments your application test suite. It adds event-firing probes at instance
method entries and logs to a human-readable JSON format. Included is an SDK with
an annotation that gives developers control over the instrumentation policy.
Complex probes have a large overhead, but record a huge amount of program state.
Simple probes simply fire events. In the future, probes placement will be based
on historical test run data and a budget.

\subsubsection{Requirements}

To build the project from source, the following tools are required:

\begin{itemize}
  \item Java Development Kit (v.7+) \cite{jdk}
  \item Android SDK (with the latest tools) \cite{androidSDK}
  \item Gradle (v.1.10+) \cite{gradle}
\end{itemize}

We recommend using IntelliJ \cite{intellij}, Android Studio \cite{androidStudio}
or some other Java-centric IDE, although anything is fine in principal.

\subsubsection{Running the Project}

To try \venera on the included Android application (skeleton-android-app), run
from the command line:

\begin{lstlisting}[numbers=none]
cd skeleton-android-app
gradle clean connectedAndroidTest -P=venera
\end{lstlisting}

IntelliJ users can create a skeleton-android-app Gradle build configuration and
run the task {\tt clean connectedAndroidTest} with script parameters {\tt
-P=venera}.

To run \venera on your own Android application, modify the {\tt build.gradle}
file under the {\tt venera-instrumentation} directory to point to your APKs.
You'll need to also provide a Gradle task (or equivalent) to pull the results
from the device in the same way that skeleton-android-app's build.gradle already
does. Note that integration with other Android applications will be made more
intuitive in the future.


\subsection{\jenkinsPlugin System Manual}

The \jenkinsPlugin maintains a HSQL database of standard test artifacts
and their associated Venera JSON logs. It will eventually be responsible for
\venera result presentation and analysis.

\subsubsection{Source Code}

The full source code is freely available at \cite{heisentestPlugin}.

\subsubsection{Requirements}

To build the project from source, the following tools are required:

\begin{itemize}
  \item Java Development Kit (v.7+) \cite{jdk}
  \item Maven (v.3+) \cite{maven}
\end{itemize}

We recommend using IntelliJ \cite{intellij} or some other Java-centric IDE,
although any editor is fine in theory.

\subsubsection{Running the Project}

From the command line:

\begin{lstlisting}[numbers=none]
mvn hpi:run
\end{lstlisting}

IntelliJ users can create a build configuration with type Maven and working
directory set to the root of the project to execute the same command.

\subsubsection{Common Problems}

When executing mvn hpi:run I get {\lq}java.net.BindException: Address already in
use{\rq}.

Something is using the TCP port (probably a previous instance that failed to
shut down correctly). On Linux, running {\tt lsof -i tcp:\$PORT} will give you
the list of processes using TCP port {\tt\$PORT}, which you can then kill with
{\tt kill \$PID}.


\subsection{Supporting Data}

\subsubsection{Full Dalvik Bytecode Example}
\label{sec:sec:full_dalvik_bytecode_example}

\begin{lstlisting}
#1              : (in Lcom/example/MainActivity;)
  name          : 'anExampleMethod'
  type          : '(Ljava/lang/String;I)Ljava/lang/String;'
  access        : 0x0002 (PRIVATE)
  code          -
  registers     : 4
  ins           : 3
  outs          : 2
  insns size    : 5 16-bit code units
024570:                  |[024570] com.example.MainActivity.anExampleMethod:(Ljava/lang/String;I)Ljava/lang/String;
024580: 7120 8503 3200   |0000: invoke-static {v2, v3}, Lcom/example/MainActivity;.appendIntToString:(Ljava/lang/String;I)Ljava/lang/String; // method@0385
024586: 0c00             |0003: move-result-object v0
024588: 1100             |0004: return-object v0
      catches   : (none)
      positions :
        0x0000 line=22
      locals    :
        0x0000 - 0x0005 reg=1 this Lcom/example/MainActivity;
        0x0000 - 0x0005 reg=2 firstArg Ljava/lang/String;
        0x0000 - 0x0005 reg=3 secondArg I
\end{lstlisting}

\subsection{Project Plan and Interim Report}

%% Save and reset the Section counter, since this is a nested LaTeX document.
\newcounter{oldSectionCounter2}
\newcounter{oldPageCounter2}

\setcounter{oldSectionCounter2}{\value{section}}
\setcounter{oldPageCounter2}{\value{page}}

\setcounter{section}{0}
%%

\title{
	Improving Diagnostic Data Gathering for Problematic System Tests:\\
	\itshape{Project Plan}
}
\author{
	Morrison Cole\\
	\texttt{zcabg19@ucl.ac.uk}
	\and
	Supervisor: Dr Earl Barr\\
	\texttt{e.barr@ucl.ac.uk}
	\and
	External Supervisor: Colin Vipurs\\
	\texttt{colin.vipurs@shazam.com}
}
\date{November 13, 2013}

\maketitle

\section{Aims}

\begin{enumerate}
	\item{
		To detect any flaky\footnote{A \flaky test is one that fails intermittently
		for no obvious reason.} tests present in a test suite, by storing and
		analysing the results of historical builds on a continuous integration
		server, so that they may be brought to the attention of developers at Shazam
		to encourage the upkeep of a stable build.
	}
	\item{
		To automatically, for some non-trivial subset of the \flaky tests in a test
		suite, extract information that speeds their resolution, either by fixing
		the test case(s) or the tested program, for a reasonable cost.
	}
\end{enumerate}

\section{Objectives}

\begin{enumerate}
	\item{
		Manually study the \flaky tests present at Shazam. Understand how these
		tests fit into the bigger picture --- consider developer workflow, number of
		tests, type of tests, code coverage, and ratio of black to white box
		testing. Find out if developers consider these tests to be a problem
		themselves.
	}
	\item{
		Develop software to:
		\begin{enumerate}
			\item{
				Pre-process JUnit-style test results from a build, so that they may be
				persisted in a database for analysis.
			}
			\item{
				Determine, over a set of test runs, which tests are \flaky. Attempt to
				detect trends --- common failure causes, etc. --- using statistical
				tools such as R and Weka.
			}
			\item{
				Produce a customisable visual representation of the detected \flaky
				tests for each build. Attach any relevant diagnostic information to each
				data point.
			}
		\end{enumerate}
	}
	\item{
		Review published research in the problem space --- namely, studies aiming to
		improve diagnostic information from runtime crashes/errors. Techniques and
		ideas developed in these projects will no doubt be useful to us.
	}
	\item{
		Develop instrumentation (using a tool such as Java MOP or ASM) to
		automatically gather information that will be useful for developers
		investigating a particular test with the aim of making it stable.
	}
\end{enumerate}

\section{Deliverables}

\begin{enumerate}
	\item{
		An overview of the state of the test suites across the teams at Shazam. As
		well as this specific case, the industry as a whole should be considered.
		The impact that \flaky tests have upon the productivity of software teams
		should be discussed.
	}
	\item{
		A tested, documented Jenkins\footnote{Jenkins is an open source continuous
		integration tool written in Java. It has a rich library of existing plugins
		and is used at Shazam, as well as at countless other commercial software
		companies.} plugin that fulfils objective 2, customized to display the data
		that Shazam wants to see. This will be deployed at Shazam as early in the
		project as possible. This will be submitted publicly to the Jenkins project
		in the hope that it is accepted to the official list of plugins.
	}
	\item{
		Literature survey describing current research in the space detailed in
		objective 3. Cover any non-trivial techniques or tools that we use that may
		be difficult for an unfamiliar reader to grasp.
	}
	\item{
		Develop a technique to instrument the \flaky tests whilst avoiding altering
		the runtime execution as much as possible. This could be done (for example)
		by selecting random samples of the complete instrumentation, running the
		tests in question multiple times, and reconstructing the complete
		instrumentation from the set of runs.
	}
\end{enumerate}

\section{Work Plan}

\begin{center}
    \begin{tabular}{ | l | p{6cm} |}
    \hline
    Period & Work \% \\ \hline
    Project start to end October (4 weeks) & Literature search and review.
    Define IP terms with Shazam. \\ \hline
    Mid-October to mid-November (4 weeks) & Literature review continued.
    Software and tools researched and familiarised. Access to Shazam's codebase
    gained. \\ \hline
    November (4 weeks) & Jenkins plugin started. Literature review continued.
    \\ \hline
    End-November to mid-January (8 weeks) & Jenkins plugin finished and deployed
    at Shazam. Work on \flaky test instrumentation started. Interim report
    written. \\ \hline
    Mid-January to mid-February (4 weeks) & Iterate on Jenkins plugin based on
    Shazam's feedback (if necessary). Continue work on \flaky test
    instrumentation. \\ \hline
    Mid-February to end of March (6 weeks) & Jenkins plugin submitted to the
    Jenkins project. \Flaky test instrumentation complete. Final report. \\
    \hline
    \end{tabular}
\end{center}

\newpage

% Restore the old value of the Section counter
\setcounter{section}{\value{oldSectionCounter2}}
\setcounter{page}{\value{oldPageCounter2}}
%%

%% Save and reset the Section counter, since this is a nested LaTeX document.
\newcounter{oldSectionCounter}
\newcounter{oldPageCounter}

\setcounter{oldSectionCounter}{\value{section}}
\setcounter{oldPageCounter}{\value{page}}

\setcounter{section}{0}
%%

\title{
	Improving Diagnostic Data Gathering for Problematic System Tests:\\
	\itshape{Interim Report}
}
\author{
	Morrison Cole\\
	\texttt{zcabg19@ucl.ac.uk}
	\and
	Supervisor: Dr Earl Barr\\
	\texttt{e.barr@ucl.ac.uk}
	\and
	External Supervisor: Colin Vipurs\\
	\texttt{colin.vipurs@shazam.com}
}
\date{January 22, 2014}

\maketitle

\section{Progress So Far}

\begin{enumerate}

	\item Reviewed existing literature in the research space. In particular, the
	work of Ben Liblit (Bug Isolation via Remote Program Sampling, Statistical
	Debugging of Sampled Programs).
	\item Negotiated a satisfactory verbal agreement with Shazam which states
	rather broadly that:
		\begin{enumerate}

			\item{\itshape they understand that we are academics who wish to publish
			certain information};
			\item {\itshape we understand that Shazam will not allow information that
			may negatively impact the business to be released publicly}.

		\end{enumerate}
	\item Received VPN credentials to access Shazam's internal Continuous
	Integration and Version Control so that the project can be worked on campus.
	\item Wrote scripts to pull both new and historical build results from
	Shazam's Continuous Integration servers. Have been running these and storing
	the data so that we can analyze it later.
	\item Set up a project Git repository / Wiki and began development of a
	Jenkins plugin. All code has been test driven from the Unit Test level so far,
	and has been written with as little coupling to Jenkins as possible. The
	plugin currently runs as a post-build step, parses all JUnit-style results and
	persists them in a database.
	\item Have begun to look at open-source projects for other examples of \flaky
	tests - Mozilla's Tinderbox Build Log in particular. Any particularly
	interesting cases have been noted on the project Wiki.

\end{enumerate}

\newpage

\section{Future Plans}

\begin{enumerate}

	\item Complete the remaining features of the Jenkins Plugin:
	\begin{enumerate}

		\item Using Weka / R, analyse the test suite historically and attempt to
		detect useful trends such as common failure causes.
		\item Produce an output Web page that displays the \flaky tests and any
		relevant diagnostic information.

	\end{enumerate}
	\item Continue gathering data on \flaky tests at Shazam and at least two other
	projects / organisations. This means collecting developer statements, finding
	examples of \flaky tests in open-source projects, documenting developer
	workflow etc.
	\item Develop a system to instrument Java code such that we share the cost of
	automatically gathering additional runtime information over multiple test-runs
	and are able to aggregate it afterwards.

\end{enumerate}

\newpage

% Restore the old value of the Section counter
\setcounter{section}{\value{oldSectionCounter}}
\setcounter{page}{\value{oldPageCounter}}
%%


\subsection{Partial Code Listing}

This section lists a significant portion of the source code that makes up
\venera. Reading code on paper is not ideal --- we suggest checkout out the full
\venera source code with {\tt git} from \cite{heisentestInstrumentation} and the
\jenkinsPlugin source code from \cite{heisentestPlugin}.

\begin{landscape}
\lstinputlisting{../venera/venera-instrumentation/src/./main/java/com/heisentest/venera/Main.java}
\lstinputlisting{../venera/venera-instrumentation/src/./main/java/com/heisentest/venera/VeneraApkProcessor.java}
\lstinputlisting{../venera/venera-instrumentation/src/./main/java/com/heisentest/venera/VeneraApkInstrumenter.java}
\lstinputlisting{../venera/venera-instrumentation/src/./main/java/com/heisentest/venera/transform/dex/VeneraDexTransformer.java}
\lstinputlisting{../venera/venera-instrumentation/src/./main/java/com/heisentest/venera/transform/dex/visitor/SimpleInstanceMethodEntryMethodVisitor.java}
\lstinputlisting{../venera/venera-instrumentation/src/./main/java/com/heisentest/venera/transform/dex/visitor/VeneraRegisterAllocatingMethodVisitor.java}
\end{landscape}
