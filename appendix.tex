\section{Appendix}
\label{sec:appendix}


\subsection{System Manual}

\begin{mdframed}
	\begin{itemize}
		\item This should include all the technical details that would enable a student to continue your project next year, and be able to amend or extend your code. For example, Where is the code? What do you do to compile and install it?
	\end{itemize}
\end{mdframed}

\subsection{User Manual}

\begin{mdframed}
	\begin{itemize}
		\item This should give enough information for someone to use what you have designed and implemented. It is a good place to include screen shots of the application or output.
	\end{itemize}
\end{mdframed}

\subsection{Supporting documentation or data}

\begin{mdframed}
	\begin{itemize}
		\item If you have additional data, diagrams or thing s like a set of use case specifications that are relevant to the documentation of your work then they can be included in an appendix section. Don't just include everything, only items that are directly relevant to the documentation or description of your work.
	\end{itemize}
\end{mdframed}

\subsubsection{Full Dalvik Bytecode Example}
\label{sec:sec:full_dalvik_bytecode_example}

\begin{lstlisting}
#1              : (in Lcom/example/MainActivity;)
  name          : 'anExampleMethod'
  type          : '(Ljava/lang/String;I)Ljava/lang/String;'
  access        : 0x0002 (PRIVATE)
  code          -
  registers     : 4
  ins           : 3
  outs          : 2
  insns size    : 5 16-bit code units
024570:                  |[024570] com.example.MainActivity.anExampleMethod:(Ljava/lang/String;I)Ljava/lang/String; (*@\label{method_body_definition}@*)
024580: 7120 8503 3200   |0000: invoke-static {v2, v3}, Lcom/example/MainActivity;.appendIntToString:(Ljava/lang/String;I)Ljava/lang/String; // method@0385
024586: 0c00             |0003: move-result-object v0
024588: 1100             |0004: return-object v0 (*@\label{method_body_end_definition}@*)
      catches   : (none)
      positions :
        0x0000 line=22
      locals    :
        0x0000 - 0x0005 reg=1 this Lcom/example/MainActivity;
        0x0000 - 0x0005 reg=2 firstArg Ljava/lang/String;
        0x0000 - 0x0005 reg=3 secondArg I
\end{lstlisting}


\subsection{Test results and test reports}

\begin{mdframed}
	\begin{itemize}
		\item If you have test results that add to the value of the report, but which would not fit within the page limit of the main report, you can include them as an appendix. Don't add them just to pad the report, though.
	\end{itemize}
\end{mdframed}

\subsection{Project Plan and Interim Report}

\newcounter{oldSectionCounter2}
\newcounter{oldPageCounter2}

\setcounter{oldSectionCounter2}{\value{section}}
\setcounter{oldPageCounter2}{\value{page}}

% Set the current value of section to 0
\setcounter{section}{0}

\title{
	Improving Diagnostic Data Gathering for Problematic System Tests:\\
	\itshape{Project Plan}
}
\author{
	Morrison Cole\\
	\texttt{zcabg19@ucl.ac.uk}
	\and
	Supervisor: Dr Earl Barr\\
	\texttt{e.barr@ucl.ac.uk}
	\and
	External Supervisor: Colin Vipurs\\
	\texttt{colin.vipurs@shazam.com}
}
\date{November 13, 2013}

\maketitle

\section{Aims}

\begin{enumerate}
	\item{
		To detect any flaky\footnote{A \flaky{} test is one that fails intermittently for no obvious reason.} tests present in a test suite, by storing and analysing the results of historical builds on a continuous integration server, so that they may be brought to the attention of developers at Shazam to encourage the upkeep of a stable build.
	}
	\item{
		To automatically, for some non-trivial subset of the \flaky{} tests in a test suite, extract information that speeds their resolution, either by fixing the test case(s) or the tested program, for a reasonable cost.
	}
\end{enumerate}

\section{Objectives}

\begin{enumerate}
	\item{
		Manually study the \flaky{} tests present at Shazam. Understand how these tests fit into the bigger picture --- consider developer workflow, number of tests, type of tests, code coverage, and ratio of black to white box testing. Find out if developers consider these tests to be a problem themselves.
	}
	\item{
		Develop software to:
		\begin{enumerate}
			\item{
				Pre-process JUnit-style test results from a build, so that they may be persisted in a database for analysis.
			}
			\item{
				Determine, over a set test runs, which tests are \flaky{}. Attempt to detect trends --- common failure causes, etc. --- using statistical tools such as R and Weka.
			}
			\item{
				Produce a customisable visual representation of the detected \flaky{} tests for each build. Attach any relevant diagnostic information to each data point.
			}
		\end{enumerate}
	}
	\item{
		Review published research in the problem space – namely, studies aiming to improve diagnostic information from runtime crashes/errors. Techniques and ideas developed in these projects will no doubt be useful to us.
	}
	\item{
		Develop instrumentation (using a tool such as Java MOP or ASM) to automatically gather information that will be useful for developers investigating a particular test with the aim of making it stable.
	}
\end{enumerate}

\section{Deliverables}

\begin{enumerate}
	\item{
		An overview of the state of the test suites across the teams at Shazam. As well as this specific case, the industry as a whole should be considered. The impact that \flaky{} tests have upon the productivity of software teams should be discussed.
	}
	\item{
		A tested, documented Jenkins\footnote{Jenkins is an open source continuous integration tool written in Java. It has a rich library of existing plugins and is used at Shazam, as well as at countless other commercial software companies.} plugin that fulfils objective 2, customized to display the data that Shazam wants to see. This will be deployed at Shazam as early in the project as possible. This will be submitted publicly to the Jenkins project in the hope that it is accepted to the official list of plugins.
	}
	\item{
		Literature survey describing current research in the space detailed in objective 3. Cover any non-trivial techniques or tools that we use that may be difficult for an unfamiliar reader to grasp.
	}
	\item{
		Develop a technique to instrument the \flaky{} tests whilst avoiding altering the runtime execution as much as possible. This could be done (for example) by selecting random samples of the complete instrumentation, running the tests in question multiple times, and reconstructing the complete instrumentation from the set of runs.
	}
\end{enumerate}

\section{Work Plan}

\begin{center}
    \begin{tabular}{ | l | p{6cm} |}
    \hline
    Period & Work \% \\ \hline
    Project start to end October (4 weeks) & Literature search and review. Define IP terms with Shazam. \\ \hline
    Mid-October to mid-November (4 weeks) & Literature review continued. Software and tools researched and familiarised. Access to Shazam's codebase gained. \\ \hline
    November (4 weeks) & Jenkins plugin started. Literature review continued. \\ \hline
    End-November to mid-January (8 weeks) & Jenkins plugin finished and deployed at Shazam. Work on \flaky{} test instrumentation started. Interim report written. \\ \hline
    Mid-January to mid-February (4 weeks) & Iterate on Jenkins plugin based on Shazam's feedback (if necessary). Continue work on \flaky{} test instrumentation. \\ \hline
    Mid-February to end of March (6 weeks) & Jenkins plugin submitted to the Jenkins project. \flaky{} test instrumentation complete. Final report. \\ \hline
    \end{tabular}
\end{center}

\newpage

% Restore the old value of the section counter
\setcounter{section}{\value{oldSectionCounter2}}
\setcounter{page}{\value{oldPageCounter2}}

\newcounter{oldSectionCounter}
\newcounter{oldPageCounter}

\setcounter{oldSectionCounter}{\value{section}}
\setcounter{oldPageCounter}{\value{page}}

% Set the current value of section to 0
\setcounter{section}{0}

\title{
	Improving Diagnostic Data Gathering for Problematic System Tests:\\
	\itshape{Interim Report}
}
\author{
	Morrison Cole\\
	\texttt{zcabg19@ucl.ac.uk}
	\and
	Supervisor: Dr Earl Barr\\
	\texttt{e.barr@ucl.ac.uk}
	\and
	External Supervisor: Colin Vipurs\\
	\texttt{colin.vipurs@shazam.com}
}
\date{January 22, 2014}

\maketitle

\section{Progress So Far}

\begin{enumerate}
	\item{
		Reviewed existing literature in the research space. In particular, the work of Ben Liblit (Bug Isolation via Remote Program Sampling, Statistical Debugging of Sampled Programs).
	}
	\item{
		Negotiated a satisfactory verbal agreement with Shazam which states rather broadly that:
		\begin{enumerate}
		\item{\itshape they understand that we are academics who wish to publish certain information};
		\item {\itshape we understand that Shazam will not allow information that may negatively impact the business to be released publicly}.
		\end{enumerate}
	}
	\item{
		Received VPN credentials to access Shazam's internal Continuous Integration and Version Control so that the project can be worked on campus.
	}
	\item{
		Wrote scripts to pull both new and historical build results from Shazam's Continuous Integration servers. Have been running these and storing the data so that we can analyze it later.
	}
	\item{
		Set up a project Git repository / Wiki and began development of a Jenkins plugin. All code has been test driven from the Unit Test level so far, and has been written with as little coupling to Jenkins as possible. The plugin currently runs as a post-build step, parses all JUnit-style results and persists them in a database.
	}
	\item{
		Have begun to look at open-source projects for other examples of flaky tests - Mozilla's Tinderbox Build Log in particular. Any particularly interesting cases have been noted on the project Wiki.
	}
\end{enumerate}

\newpage

\section{Future Plans}

\begin{enumerate}
	\item{
		Complete the remaining features of the Jenkins Plugin:
	}
	\begin{enumerate}
		\item{
			Using Weka / R, analyse the test suite historically and attempt to detect useful trends such as common failure causes.
		}
		\item{
			Produce an output Web page that displays the flaky tests and any relevant diagnostic information.
		}
	\end{enumerate}
	\item{
		Continue gathering data on flaky tests at Shazam and at least two other projects / organisations. This means collecting developer statements, finding examples of flaky tests in open-source projects, documenting developer workflow etc.
	}
	\item{
		Develop a system to instrument Java code such that we share the cost of automatically gathering additional runtime information over multiple test-runs and are able to aggregate it afterwards.
	}
\end{enumerate}

\newpage

% Restore the old value of the section counter
\setcounter{section}{\value{oldSectionCounter}}
\setcounter{page}{\value{oldPageCounter}}

\subsection{Code Listing}

\begin{mdframed}
	\begin{itemize}
		\item Your code should be properly presented and formatted neatly. Don't let long lines of code arbitrarily wrap round to the next line, as this looks very messy. In order not to use up too many pages you can switch the paper orientation to lanscape and print the code in two columns.

		In general, don't add more than about 25 pages of code listing to the report. If your code does not fit within 25 pages, you can provide a listing of the most interesting parts of your code (but include around 20-25 pages) or parts of the code you may need to reference from the main chapters. If you don't include everything, it must be clear that this is not the complete listing. State which parts you've included, add a brief explanation of why you've included these particular parts, and provide a brief summary of which code you have omitted.
	\end{itemize}
\end{mdframed}

\subsubsection{Source Code Repositories}

The full source code listings are available at \cite{heisentestInstrumentation} and \cite{heisentestPlugin}.
