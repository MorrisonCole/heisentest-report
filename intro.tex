% This chapter is relatively short (2-4 pages) and must leave the reader very clear on what the project is about and what your goals are.
\section{Introduction}
\label{sec:intro}

\begin{framed}
	\begin{itemize}
		\item Outline the problem you are working on, why it is interesting and what the challenges are.
		\item List your aims and goals. An aim is something you intend to achieve (i.e., learn a new programming language and apply it in solving the problem), while a goal is something specific you expect to deliver (e.g., a working application with a particular set of features).
		\item Give an overview of how you carried out your project (e.g., an iterative approach).
		\item A brief overview of the rest of the chapters in the report (a guide to the reader of the overall structure of the report).
	\end{itemize}
\end{framed}

\textit{The following are placeholders; subsections are to be removed.}

\subsection{Testing is used in practice across the industry}

Testing is as much a part of modern software development as deployment. Enforces quality, prevents regressions, reduces need for manual testing in certain areas. In the age of continuous deployment, developers have turned to automated testing to ensure that their software does not suffer regressions or recurring bugs. A team might deploy software into a production environment multiple times per week. It is essential to write and maintain a suite of unit, integration and system tests to have the confidence to do so.

\subsection{Testing is flawed/not always reliable}

Flaky tests across the industry.

\subsection{What we intend to do}

Apply adaptive bug isolation techniques to testing to gather information on flaky test cases with the aim of aiding their stabilization.

\subsection{How the rest of the paper is structured}