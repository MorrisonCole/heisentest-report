%% Save and reset the Section counter, since this is a nested LaTeX document.
\newcounter{oldSectionCounter2}
\newcounter{oldPageCounter2}

\setcounter{oldSectionCounter2}{\value{section}}
\setcounter{oldPageCounter2}{\value{page}}

\setcounter{section}{0}
%%

\title{
	Improving Diagnostic Data Gathering for Problematic System Tests:\\
	\itshape{Project Plan}
}
\author{
	Morrison Cole\\
	\texttt{zcabg19@ucl.ac.uk}
	\and
	Supervisor: Dr Earl Barr\\
	\texttt{e.barr@ucl.ac.uk}
	\and
	External Supervisor: Colin Vipurs\\
	\texttt{colin.vipurs@shazam.com}
}
\date{November 13, 2013}

\maketitle

\section{Aims}

\begin{enumerate}
	\item{
		To detect any flaky\footnote{A \flaky test is one that fails intermittently
		for no obvious reason.} tests present in a test suite, by storing and
		analysing the results of historical builds on a continuous integration
		server, so that they may be brought to the attention of developers at Shazam
		to encourage the upkeep of a stable build.
	}
	\item{
		To automatically, for some non-trivial subset of the \flaky tests in a test
		suite, extract information that speeds their resolution, either by fixing
		the test case(s) or the tested program, for a reasonable cost.
	}
\end{enumerate}

\section{Objectives}

\begin{enumerate}
	\item{
		Manually study the \flaky tests present at Shazam. Understand how these
		tests fit into the bigger picture --- consider developer workflow, number of
		tests, type of tests, code coverage, and ratio of black to white box
		testing. Find out if developers consider these tests to be a problem
		themselves.
	}
	\item{
		Develop software to:
		\begin{enumerate}
			\item{
				Pre-process JUnit-style test results from a build, so that they may be
				persisted in a database for analysis.
			}
			\item{
				Determine, over a set of test runs, which tests are \flaky. Attempt to
				detect trends --- common failure causes, etc. --- using statistical
				tools such as R and Weka.
			}
			\item{
				Produce a customisable visual representation of the detected \flaky
				tests for each build. Attach any relevant diagnostic information to each
				data point.
			}
		\end{enumerate}
	}
	\item{
		Review published research in the problem space --- namely, studies aiming to
		improve diagnostic information from runtime crashes/errors. Techniques and
		ideas developed in these projects will no doubt be useful to us.
	}
	\item{
		Develop instrumentation (using a tool such as Java MOP or ASM) to
		automatically gather information that will be useful for developers
		investigating a particular test with the aim of making it stable.
	}
\end{enumerate}

\section{Deliverables}

\begin{enumerate}
	\item{
		An overview of the state of the test suites across the teams at Shazam. As
		well as this specific case, the industry as a whole should be considered.
		The impact that \flaky tests have upon the productivity of software teams
		should be discussed.
	}
	\item{
		A tested, documented Jenkins\footnote{Jenkins is an open source continuous
		integration tool written in Java. It has a rich library of existing plugins
		and is used at Shazam, as well as at countless other commercial software
		companies.} plugin that fulfils objective 2, customized to display the data
		that Shazam wants to see. This will be deployed at Shazam as early in the
		project as possible. This will be submitted publicly to the Jenkins project
		in the hope that it is accepted to the official list of plugins.
	}
	\item{
		Literature survey describing current research in the space detailed in
		objective 3. Cover any non-trivial techniques or tools that we use that may
		be difficult for an unfamiliar reader to grasp.
	}
	\item{
		Develop a technique to instrument the \flaky tests whilst avoiding altering
		the runtime execution as much as possible. This could be done (for example)
		by selecting random samples of the complete instrumentation, running the
		tests in question multiple times, and reconstructing the complete
		instrumentation from the set of runs.
	}
\end{enumerate}

\section{Work Plan}

\begin{center}
    \begin{tabular}{ | l | p{6cm} |}
    \hline
    Period & Work \% \\ \hline
    Project start to end October (4 weeks) & Literature search and review.
    Define IP terms with Shazam. \\ \hline
    Mid-October to mid-November (4 weeks) & Literature review continued.
    Software and tools researched and familiarised. Access to Shazam's codebase
    gained. \\ \hline
    November (4 weeks) & Jenkins plugin started. Literature review continued.
    \\ \hline
    End-November to mid-January (8 weeks) & Jenkins plugin finished and deployed
    at Shazam. Work on \flaky test instrumentation started. Interim report
    written. \\ \hline
    Mid-January to mid-February (4 weeks) & Iterate on Jenkins plugin based on
    Shazam's feedback (if necessary). Continue work on \flaky test
    instrumentation. \\ \hline
    Mid-February to end of March (6 weeks) & Jenkins plugin submitted to the
    Jenkins project. \Flaky test instrumentation complete. Final report. \\
    \hline
    \end{tabular}
\end{center}

\newpage

% Restore the old value of the Section counter
\setcounter{section}{\value{oldSectionCounter2}}
\setcounter{page}{\value{oldPageCounter2}}
%%
