% This chapter is typically 204 pages long but could be longer if the project work requires more extensive evaluation.
\section{Conclusion}
\label{sec:conc}

\begin{mdframed}
	\begin{itemize}
		\item A summary of what the project has achieved. Address each goal set out in the introduction.
		\item A critical evaluation of the results of the project (e.g., how well were the goals met, is the application fit for purpose, has good design and implementation practice been followed, was the right implementation technology chosen and so on).
		\item Future work. How could the project be developed if you had another 6 months.
		\item Wrap-up and final thoughts on your project.
	\end{itemize}
\end{mdframed}


\subsection{Design Failures}

Could have used Soot to convert Dalvik bytecode into standard Java .class files using Dexpler \cite{bartel:soap2012} and instrumented from then on. However, the DEX code output by Dexpler is often invalid, as is evident from the bug reports and ongoing development at GitHub. Perhaps in the future, Soot would be a more viable option. Certainly the features it boasts are attractive for the Heisentest project.


\subsection{Future Work}

\subsubsection{Gathering data from local test runs}

In a typical test driven development cycle, tests are run constantly --- both by the developers and the continuous integration build agents. We could gather test run data from the tests run by developers themselves by sending the results over the network to the Splatter CI plugin.

\begin{itemize}
	\item This would require {\lq}noise{\rq} reduction --- developers break tests locally often, we wouldn't want to adjust our flakiness values based on changes that never make it onto the development branch.
	\item It should be easy to {\lq}opt-out{\rq} of this --- our instrumentation does affect build times and developers may not appreciate the extra overhead!
\end{itemize}

\subsubsection{Commit blaming}

Our CI plugin could display an estimated commit {\lq}blame{\rq} range, along with relevant build numbers if a previously stable test is suddenly observed to be \flaky{}.

\subsubsection{Retiring trusted tests}

Projects with a huge number of tests face lengthy build times. It may be in a teams' interest to remove trusted tests from the per-commit build and instead run them on a seperate cycle to tighten the immediate feedback loop. We could provide a listing of {\lq}most trusted{\rq} tests (\ie, tests that have exhibited no flakiness over a significant period).

\subsubsection{Porting the plugin}

Jenkins is but one of many available continuous integration tools \cite{ContinuousIntegrationSoftware}. Hudson, TeamCity, CruiseControl and Team Foundation Server are amongst the notable alternatives. A Hudson port would be especially trivial, since the Jenkins codebase is still fairly similar. In fact, the code-bases have not diverged significantly since 2010 when Jenkins materialised.

\todo{The instrumentation does not have to be run locally. Indeed, this may damage the inferred data anyway (especially if there are common but obvious failures which are then fixed without issue). Instead, it should be run with the standard build or on it’s own dedicated job (e.g. an overnight build).}
