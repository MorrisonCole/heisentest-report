% This chapter should cover background information, related work, research done and tools or software selected for use in the project.
% You should not include well-known things (e.g., HTML or Java) or try to give tutorials on how to use a tool or code library (use references to books and websites for that information). Everything you include should be directly relevant to your work and the relationship made clear.
% This chapter is likely to be fairly substantial, perhaps 8-10 pages.
\section{Context}
\label{sec:context}

\begin{framed}
	\begin{itemize}
		\item Provide the necessary context and background information to describe how your project relates to what is already known or available.
		\item If relevant, a survey of similiar solutions, programs or applications to yours, and how yours is differentiated.
		\item A description of the research carried out to learn about the nature of the problem(s) being investigated and potential solutions. The form of the research will vary widely depending on the kind of project. For example, it might involve searching through research publications and online resources, or might involve an exploration of design possibilities for a user interface or program structure.
		\item Outline and reference the sources of information you are drawing on (papers, books, websites, etc.). State how each relates to your work.
		\item Introduce the software, programming languages, library code, frameworks and other tools that you are using. Discuss choices and make clear which you made use of and why.
	\end{itemize}
\end{framed}

Test driven development, when taken to its limits, works like this:

\begin{itemize}
	\item A developer receives a specification for a feature (this could be a user story, given when thens i.e. BDD-style - cucumber, or any number of different presentations)

	\item The developer writes a system test to ensure the feature works. System tests operate at the system level, so they often read like BDD requirements, e.g.:

	\begin{verbatim}
  	Scenario: Add to Basket
  	  Given I am viewing the store page
  	    And I have no items in my basket
  	  When I add an item to my basket
  	  Then that item is shown in my basket
	\end{verbatim}

	Could be eventually converted in to the following Java code:

	\lstinputlisting[language=Java]{bddExample.java}

	\item The developer runs/edits the test until it fails for the right reason.

	\item The developer begins implementing the feature - again writing tests first, but this time on the finer levels of unit / integration.

	\item When all tests pass, the feature is completed.
\end{itemize}

Note that there are many points for interpretation along the steps of this process. For example, an interface may be tested for the presence of a specific image, but its location may be ignored. Because of this, manual spot checks and alpha / beta staged releases are often used.

In order for a test suite to improve a team’s work, it must be trustworthy. Tests must be written carefully to ensure that they fail when the code under test is indeed broken, and pass otherwise. As is often the case, when tests fail unexpectedly, they are removed from the suite until they are fixed.

Unfortunately, tests running at the system level are prone to intermittent failure.

When a subset of tests fail randomly each time the suite is run, it is all too easy for developers to begin ignoring real failures if and when they do occur.
