% This chapter should cover background information, related work, research done and tools or software selected for use in the project.
% You should not include well-known things (e.g., HTML or Java) or try to give tutorials on how to use a tool or code library (use references to books and websites for that information). Everything you include should be directly relevant to your work and the relationship made clear.
% This chapter is likely to be fairly substantial, perhaps 8-10 pages.
\section{Context}
\label{sec:context}

\begin{itemize}
	\item Provide the necessary context and background information to describe how your project relates to what is already known or available.
	\item If relevant, a survey of similiar solutions, programs or applications to yours, and how yours is differentiated.
	\item A description of the research carried out to learn about the nature of the problem(s) being investigated and potential solutions. The form of the research will vary widely depending on the kind of project. For example, it might involve searching through research publications and online resources, or might involve an exploration of design possibilities for a user interface or program structure.
	\item Outline and reference the sources of information you are drawing on (papers, books, websites, etc.). State how each relates to your work.
	\item Introduce the software, programming languages, library code, frameworks and other tools that you are using. Discuss choices and make clear which you made use of and why.
\end{itemize}
