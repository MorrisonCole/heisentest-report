\label{sec:abstract}

% No more than half a page and typically consists of three short paragraphs.
\begin{abstract}

% \begin{mdframed}
% 	\begin{itemize}
% 		\item What the project is about, the principle aims and goals, and specific challenges.
% 		\item How you carried out the project and what work it involved.
% 		\item The results and achievements of the project.
% 	\end{itemize}
% \end{mdframed}

A flaky test case is one whose outcome is sensitive to some unknown input. We present a framework for identifying and accelerating the resolution of flaky test cases in a test suite.

The framework operates on a continuous lifecycle. To begin, each test's probability of failure is calculated. Then, each test is adaptively instrumented with respect to a budget --- state-logging probes are placed in choice locations. Finally, aggregated data is analysed to identify predicates strongly associated with test failure. At each stage, the framework can work autonomously or with input from a developer.

\todo{
	We present a case study of the Android team and their codebase at Shazam.

	We also formulate some contextual research questions:

	\begin{itemize}
		\item Why are flaky tests a problem?
		\item What are the costs of a flaky test?
	\end{itemize}
}

Our approach applies bug isolation techniques in an environment free from previously enfeebling characteristics, allowing us to record a rich set of program state.

We discuss and document the implementation of a working proof of concept and evaluate its application at a modern startup, Shazam.

\end{abstract}
